\documentclass[a0paper,portrait]{baposter}
\usepackage[utf8]{inputenc}
\usepackage{cmap}
\usepackage[T1]{fontenc}
\usepackage[english]{babel}
\usepackage{biblatex}
\usepackage{enumitem}
\addbibresource{references.bib}

\usepackage{url}
\usepackage{breakurl}

\usepackage{amsmath}
\usepackage{amsfonts}
\usepackage{amssymb}
\usepackage{amsthm}

\usepackage{graphicx}
\usepackage{tikz}
\usepackage{pgfplots}
\pgfplotsset{compat=newest,compat/show suggested version=false}

\usepackage{textcomp}
\usepackage{listings}

\usepackage{lmodern}
\usepackage{mathrsfs}

\usepackage{multirow}

 \usepackage{unicamplogo}
 \usepackage{imecclogo}

\definecolor{bordercol}{RGB}{0,0,150}
\definecolor{headercol1}{RGB}{131,137,220}
\definecolor{headercol2}{RGB}{192,255,255}
\definecolor{headerfontcol}{RGB}{0,0,0}
\definecolor{boxcolor}{RGB}{220,240,240} 
\definecolor{bgcolor1}{RGB}{255,255,180}
\definecolor{bgcolor2}{RGB}{255,180,100}

\begin{document}
\begin{poster}
  {
    grid=false,
    columns=2,
    headerfont=\Large\sf\bf,
    background=shadeTB,
    borderColor=bordercol,
    headerColorOne=headercol1,
    headerColorTwo=headercol2,
    headerFontColor=headerfontcol,
    boxColorOne=boxcolor,
    boxheaderheight=1.8em,
    bgColorOne=bgcolor1,
    bgColorTwo=bgcolor2,
    textborder=roundedsmall,
    headerborder=closed,
    headershape=smallrounded,
    boxshade=plain
  }
  {
 \unicamplogol
  }
  {
    perprof-py \vspace{2pt} \\ A Python Package for Performance Profile
  }
  {
    \begin{tabular}{ccc}
      Raniere Gaia Costa da Silva &
      Abel Soares Siqueira &
      Luiz-Rafael Santos \\
      \url{raniere@ime.unicamp.br} &
      \url{abel@ime.unicamp.br} &
      \url{lrsantos11@gmail.com}
    \end{tabular}
  }
  {
 %   \imecclogo
  }

  \linespread{1.1}

  \begin{posterbox}[column=0]{Introduction}
    Benchmarking optimization packages is very important in the optimization
    field, not only because it is one of the ways to compare solvers, but also
    to uncover deficiencies that could be overlooked. During benchmarking, one
    can obtain several informations, like CPU time, number of function
    evaluations, number of iterations and so on. These informations, if
    presented as tables, can be difficult to be analyzed, due, for instance,
    to large amounts of data. Therefore, researchers started testing tools to
    better process and understand this data.  One of the most widespread ways
    to do so is by using Performance Profile graphics as proposed by
    \textcite{Dolan2002}.
  \end{posterbox}

  \begin{posterbox}[column=0,below=auto]{Performance Profile}
    The performance profile of a solver is the cumulative distribution
    function of a performance metric, e.g., CPU time, number of functions
    evaluations, number of iterations, and so on, that we will call \emph{cost}.

    Given a set $P$ of problems and a set $S$ of solvers, for each problem $p
    \in P$ and solver $s \in S$, we define $t_{ps}$ as the cost
    required to solve problem $p$ by solver $s$ and
    \begin{align*}
      r_{ps} = \frac{t_{ps}}{\min\{t_{ps}: s \in S\}}
    \end{align*}
    as the performance ratio of solver $s$ for the problem $p$ when compared
    with the best performance by any solver on this problem.

    \textcite{Dolan2002} define
    \begin{align*}
      \rho_s(\tau) = \frac{| \{p \in P: r_{ps} \leq \tau\} |}{| P |}
    \end{align*}
    where $\rho_s(\tau)$ is the probability for solver $s \in S$ to solve one
    problem within a factor $\tau \in \mathbb{R}$ of the best performance
    ratio. For a given $\tau$, the best solver is the one with the highest
    value for $\rho_s(\tau)$.

    \includegraphics[width=0.95\linewidth]{abc-semilog.pdf}
  \end{posterbox}

  \begin{posterbox}[column=0,below=auto,name=perprof]{perprof-py}
    We implemented a free/open software that generates performance profiles using data
    provided by the user in a friendly manner. This software produces graphics in PDF
    using LaTeX with PGF/TikZ and pgfplots packages,
    in PNG using matplotlib and can also be easily extended to
    use other plot libraries. The software is implemented in Python3 with
    support for internationalization and is available at
    \url{https://github.com/lpoo/perprof-py}.
  \end{posterbox}

  \begin{posterbox}[column=1]{Features}
    Some of perprof-py features are:
    \begin{itemize}[noitemsep]
      \item high quality output (in more than one format);
      \item lots of native options/filters;
      \item input processing to limit solver minimum and maximum costs and set
        of problems;
      \item native internationalization support.
    \end{itemize}

    \begin{center}
      \begin{tabular}{cc}
        \includegraphics[scale=0.4]{abc-bw-semilog.pdf} &
        \includegraphics[scale=0.4]{abc.pdf} \\
        Black and white & Linear scale \\
        \includegraphics[scale=0.4]{abc-100.pdf} &
        \includegraphics[scale=0.4]{abc-hs} \\
        Performance ratio limit & Subset (Hock-Schittkowski)
      \end{tabular}
    \end{center}

    Besides the previous features, the package can be easily used in a range
    of external tools that support Python.
  \end{posterbox}

  \begin{posterbox}[column=1,below=auto]{Input file}
    Each solver to be used in the benchmark must have a file like:

    \begin{lstlisting}
#Name Solver
Problem01 exit01 time01
Problem02 exit02 time02
    \end{lstlisting}

    Each line other than the solver name has 3 columns that mean, in order:
    \begin{itemize}[noitemsep]
      \item The name of the problem;
      \item Exit flag;
      \item Elapsed time.
    \end{itemize}
  \end{posterbox}

  \begin{posterbox}[column=1,below=auto]{Future works}
    Some of our next improvements consist of:
    \begin{itemize}[noitemsep]
      \item add metadata support and also support data profiles as defined by
        \textcite{More2009};
      \item add support for the analysis of the objective value, and the primal and
        dual infeasibilities;
      \item add more backends and more output formats.
    \end{itemize}
  \end{posterbox}

  \begin{posterbox}[column=1,below=auto]{References}
    \printbibliography[heading=none]
  \end{posterbox}

  \begin{posterbox}[column=0,span=2,below=auto,height=bottom,
    boxshade=none,textborder=none,headerborder=none,headershade=plain,
  headerColorOne=bgcolor2, boxheaderheight=0cm]{}
  \begin{center}
    \begin{tabular}{cl}
      \multirow{3}{*}{\includegraphics[height=24pt]{figures/cc-by}} & \\
                                                                    & \Large
      This work is licensed under a
       Creative Commons Attribution 3.0 Unported License.
    \end{tabular}
  \end{center}
  \end{posterbox}
\end{poster}
\end{document}
