\documentclass[]{beamer}
\usepackage[utf8]{inputenc}
\usepackage{cmap}
\usepackage[T1]{fontenc}
\usepackage[english]{babel}
\usepackage{beamerposter}
\usepackage{geometry}
\geometry{paperwidth=90cm,paperheight=100cm}
\usepackage{biblatex}
\addbibresource{references.bib}

\usepackage{url}
\usepackage{breakurl}
\usepackage{hyperref}

\usepackage{amsmath}
\usepackage{amsfonts}
\usepackage{amssymb}
\usepackage{amsthm}

\usepackage{graphicx}
\usepackage{tikz}
\usepackage{pgfplots}

%%%%%%%%%%%%%%%%%%%%%%%%%%%%%% Pacotes: tabelas %%%%%%%%%%%%%%%%%%%%%%%%%%%%%%
\usepackage{multicol}
\usepackage{multirow}

%%%%%%%%%%%%%%%%%%%%%%%%%%%%% Pacotes: algoritmos %%%%%%%%%%%%%%%%%%%%%%%%%%%%%
\usepackage{algorithmic}
\usepackage{algorithm}
\floatname{algorithm}{Algoritmo}
\renewcommand{\listalgorithmname}{Lista de Algoritmos}


%%%%%%%%%%%%%%%%%%%%%%%%%%%%%% Pacotes: códigos %%%%%%%%%%%%%%%%%%%%%%%%%%%%%%
\usepackage{textcomp}
\usepackage{listings}
\renewcommand\lstlistingname{Código}
\renewcommand\lstlistlistingname{Lista de Códigos}


%%%%%%%%%%%%%%%%%%%%%%%%%%%%%%% Pacotes: index %%%%%%%%%%%%%%%%%%%%%%%%%%%%%%%
\usepackage{makeidx}
\makeindex


%%%%%%%%%%%%%%%%%%%%%%%%%%%%%%% Pacotes: fontes %%%%%%%%%%%%%%%%%%%%%%%%%%%%%%
\usepackage{lmodern}
\usepackage{mathrsfs}


%%%%%%%%%%%%%%%%%%%%%%%%%%%%%%% Início do poster %%%%%%%%%%%%%%%%%%%%%%%%%%%%%%%
\usetheme{Copenhagen}
\setbeamertemplate{headline}{}
\setbeamertemplate{footline}{}
\begin{document}
\begin{frame}[t,fragile]
  \begin{center}
    \begin{huge}
      perprof-py -- A Python Package for Performance Profile 
    \end{huge}

    \vspace{40pt}
    \begin{Large}
      \begin{tabular}[]{c}
        Raniere Gaia Costa da Silva \\
        \url{raniere@ime.unicamp.br}
      \end{tabular} \hspace{5cm}
      \begin{tabular}[]{c}
        Abel Soares Siqueira \\
        \url{abel.s.siqueira@gmail.com}
      \end{tabular} \hspace{5cm}
      \begin{tabular}[]{c}
        Luiz Rafael dos Santos \\
        \url{lrsantos11@gmail.com}
      \end{tabular}
    \end{Large}
  \end{center}
  \vspace{40pt}

%%%%%%%%%%%%%%%%%%%%%%%%%%%%%%% Corpo do poster %%%%%%%%%%%%%%%%%%%%%%%%%%%%%%%
% O ambiente columns é definido na classe beamer.
  \begin{columns}[t]
    \begin{column}{0.4\textwidth}
      \begin{block}{Abstract}
        Benchmarking optimization packages are very important in optimization
        field, not only because it is one of the way to compare packages, but also
        to uncover deficiencies that could be overlooked. During benchmarking, one
        can obtain several informations, like CPU time, number of functions
        evaluations, number of iterations and so on. These informations, if
        presented as tables, can be difficult to be analyzed, due, for instance,
        to large amount of data. Therefore, researchers started testing tools to
        better process and understand this data.  One of the most widespread ways
        to do so is using Performance Profile graphics proposed by
        \citeauthor{Dolan2001}.

        In this context, we implemented a free software that makes Performance
        Profile using data provided by user in a friendly manner. This software
        produces graphics in PDF using LaTeX with PGF/TikZ\nocite{TikZ} and
        pgfplots\nocite{pgfplots} packages, in PNG using
        matplotlib\nocite{Hunter:2007} and can also be easily extended to use with
        other plot library. The software is implemented in Python3 with support
        for internationalization and is available on
        \url{https://github.com/abelsiqueira/perprof-py}.
      \end{block}

      \begin{block}{Performance Profile}
        Given a set $P$ of problems and a set $S$ of solvers. For each problem $p
        \in P$ and solver $s \in S$, we define $t_{ps}$ as the computing time
        required to solve problem $p$ by solver $s$ and
        \begin{align*}
          r_{ps} = \frac{t_{ps}}{\min\{t_{ps}: s \in S\}}
        \end{align*}
        as the performance ratio of solver $s$ for the problem $p$ when compared
        with the best performance by any solver on this problem.

        \citeauthor{Dolan2001} define
        \begin{align*}
          \rho_s(\tau) = \frac{| \{p \in P: r_{ps} \leq \tau\} |}{| P |}
        \end{align*}
        where $\rho_s(\tau)$ is the probability for solver $s \in S$ to solve one
        problem within a factor $\tau \in \mathbb{R}$ of the best performance
        ratio. For a given $\tau$, the best solver is the one with the higher
        value for $\rho_s(\tau)$.

        One of the advantages of using performance profiles is the easily graphic
        analyse, it be relatively insensitive to changes in results on a small
        number of problems and also largely unaffected by small changes in
        results over many problems.
      \end{block}
    \end{column}
    \begin{column}{0.4\textwidth}
      \begin{block}{Input file}
        For every solver to be included in the benchmark there must be a file like
        the sample below:

        \begin{lstlisting}
        # Solver name
        Problem01 exit01 time01 objective01 primal01 dual01
        Problem02 exit02 time02 objective02 primal02 dual02
        Problem03 exit03 time03 objective03 primal03 dual03
        \end{lstlisting}

        Each file has 6 columns that means, in order:
        \begin{itemize}
          \item The name of the problem;
          \item Exit flag;
          \item Time spent;
          \item Objective function value;
          \item Primal infeasibility;
          \item Dual infeasibility.
        \end{itemize}
      \end{block}

      \begin{block}{Output}

      \input{abc-semilog}
      \end{block}

      \textbf{References}

      \printbibliography
    \end{column}
  \end{columns}
  \vfill
  \begin{center}
    \begin{tabular}[]{cc}
      \multirow{2}{*}{\includegraphics[height=60pt]{figures/cc-by}} &
      \Large{This work is licensed under a}\\ &
      \Large{Creative Commons Attribution 3.0 Unported License.}
    \end{tabular}
  \end{center}
\end{frame}
\end{document}
