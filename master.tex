\documentclass[]{beamer}
\usepackage[utf8]{inputenc}
\usepackage{cmap}
\usepackage[T1]{fontenc}
\usepackage[english]{babel}
\usepackage{beamerposter}
\usepackage{geometry}
\geometry{paperwidth=90cm,paperheight=100cm}
\usepackage{biblatex}
\addbibresource{references.bib}

\usepackage{url}
\usepackage{breakurl}
\usepackage{hyperref}

\usepackage{amsmath}
\usepackage{amsfonts}
\usepackage{amssymb}
\usepackage{amsthm}

\usepackage{graphicx}
\usepackage{tikz}

%%%%%%%%%%%%%%%%%%%%%%%%%%%%%% Pacotes: tabelas %%%%%%%%%%%%%%%%%%%%%%%%%%%%%%
\usepackage{multicol}
\usepackage{multirow}

%%%%%%%%%%%%%%%%%%%%%%%%%%%%% Pacotes: algoritmos %%%%%%%%%%%%%%%%%%%%%%%%%%%%%
\usepackage{algorithmic}
\usepackage{algorithm}
\floatname{algorithm}{Algoritmo}
\renewcommand{\listalgorithmname}{Lista de Algoritmos}


%%%%%%%%%%%%%%%%%%%%%%%%%%%%%% Pacotes: códigos %%%%%%%%%%%%%%%%%%%%%%%%%%%%%%
\usepackage{textcomp}
\usepackage{listings}
\renewcommand\lstlistingname{Código}
\renewcommand\lstlistlistingname{Lista de Códigos}


%%%%%%%%%%%%%%%%%%%%%%%%%%%%%%% Pacotes: index %%%%%%%%%%%%%%%%%%%%%%%%%%%%%%%
\usepackage{makeidx}
\makeindex


%%%%%%%%%%%%%%%%%%%%%%%%%%%%%%% Pacotes: fontes %%%%%%%%%%%%%%%%%%%%%%%%%%%%%%
\usepackage{lmodern}
\usepackage{mathrsfs}


%%%%%%%%%%%%%%%%%%%%%%%%%%%%%%% Início do poster %%%%%%%%%%%%%%%%%%%%%%%%%%%%%%%
\begin{document}
\begin{frame}[t,fragile]
  \begin{center}
    \begin{huge}
      perprof-py -- A Python Package for Performance Profile 

      \vspace{20pt}
      Instituto de Matemática, Estatística e Computação Científica
    \end{huge}

    \vspace{20pt}
    \begin{Large}
      \begin{tabular}[]{c}
        Raniere Gaia Costa da Silva \\
        \url{raniere@ime.unicamp.br}
      \end{tabular} \hspace{5cm}
      \begin{tabular}[]{c}
        Abel Soares Siqueira \\
        \url{abel.s.siqueira@gmail.com}
      \end{tabular} \hspace{5cm}
      \begin{tabular}[]{c}
        Abel Soares Siqueira \\
        \url{lrsantos11@gmail.com}
      \end{tabular}
    \end{Large}
  \end{center}
  \vspace{20pt}

%%%%%%%%%%%%%%%%%%%%%%%%%%%%%%% Corpo do poster %%%%%%%%%%%%%%%%%%%%%%%%%%%%%%%
% O ambiente columns é definido na classe beamer.
  \begin{columns}[t]
    \begin{column}{0.4\textwidth}
      Benchmarking optimization packages are very important in optimization
      field, not only because it is one of the way to compare packages, but also
      to uncover deficiencies that could be overlooked. During benchmarking, one
      can obtain several informations, like CPU time, number of functions
      evaluations, number of iterations and so on. These informations, if
      presented as tables, can be difficult to be analyzed, due, for instance,
      to large amount of data. Therefore, researchers started testing tools to
      better process and understand this data.  One of the most widespread ways
      to do so is using Performance Profile graphics proposed by
      \citeauthor{Dolan2001}.

      In this context, we implemented a free software that makes Performance
      Profile using data provided by user in a friendly manner. This software
      produces graphics in PDF using LaTeX with PGF/TikZ\nocite{TikZ} and
      pgfplots\nocite{pgfplots} packages, in PNG using
      matplotlib\nocite{Hunter:2007} and can also be easily extended to use with
      other plot library. The software is implemented in Python3 with support
      for internationalization and is available on
      \url{https://github.com/abelsiqueira/perprof-py}.
    \end{column}
    \begin{column}{0.4\textwidth}
      \printbibliography
    \end{column}
  \end{columns}
  \vfill
  \begin{center}
    \begin{tabular}[]{cc}
      \multirow{2}{*}{\includegraphics[height=60pt]{figures/cc-by}} &
      \Large{This work is licensed under a}\\ &
      \Large{Creative Commons Attribution 3.0 Unported License.}
    \end{tabular}
  \end{center}
\end{frame}
\end{document}
